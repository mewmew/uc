\subsection{Validation}
\label{sec:irgen_validation}

As the LLVM IR generated by the \texttt{uclang} compiler has been intentionally kept very close to the LLVM IR generated by Clang for a given uC program, testing the implementation of the IR generator becomes almost trivial. The test cases simply use Clang to generate LLVM IR from input C source files, each covering a specific language features, such as while loops or array index expressions. Since Clang outputs a lot of metadata, which is not strictly related to the semantics of the program, a post-processing script has been developed which removes any unnecessary metadata, function attributes, comments or other LLVM IR language constructs which are not required to run the test programs. The post-processed LLVM IR assembly output from Clang is then compared against the LLVM IR generated by the \texttt{uclang} IR generation algorithm, and the test case passes if the outputs are identical and it fails otherwise.

To the best of our knowledge, the LLVM IR generated by \texttt{uclang} should be valid for all possible uC programs, with one exception, array index expressions and array references in functions taking arrays as input parameters. That being said, we have discovered a wide range of issues with the generation algorithm while developing it, and have added specific test cases to ensure that fixed bugs are not reintroduced at a later point of development.

To run the test cases of the IR generation library, invoke the following command.

\begin{verbatim}
$ go test github.com/mewmew/uc/irgen
\end{verbatim}

The input source files used to validate the IR generation algorithm may be located in the \texttt{testdata/extra/irgen} directory of the \texttt{uc} repository.

\subsubsection{Example Output}

The listing example illustrates the LLVM IR output (see listing \ref{lst:r06.ll}) generated by \texttt{uclang} when compiling the Fibonacci test case \texttt{testdata/quiet/rtl/r06.c} (see listing \ref{lst:quiet/rtl/r06.c}). Note, the \texttt{r06.c} source file has been slightly modified, to commend out the preprocessing include directive and to include a return statement from main, returning the value calculated by the \texttt{fib} function as a status code from the progam.

\lstinputlisting[style=c,language=C,caption=quiet/rtl/r06.c,label=lst:quiet/rtl/r06.c]{inc/sections/6_ir_generation/r06.c}

\lstinputlisting[style=c,language=llvm,caption=r06.ll,label=lst:r06.ll]{inc/sections/6_ir_generation/r06.ll}

To compile test \texttt{r06.c} source file and run the resulting LLVM IR assembly code, invoke the following commands.

\begin{verbatim}
$ uclang -o r06.ll r06.c
$ lli r06.ll ; echo $?
\end{verbatim}

Executing the above command will print the status code \texttt{120} to standard output, thus confirming that the fifth Fibonacci number was indeed calculated correctly.

%\subsubsection{Correct RTL Test Cases}

%For listings, see Appendix~\ref{app:irgen/correct}: \nameref{app:irgen/correct}:
%\input{inc/uclang/quiet/rtl/listref.tex}

%\subsubsection{Extra Test Cases}

%For listings, see Appendix~\ref{app:irgen/extra}: \nameref{app:irgen/extra}:
%\input{inc/uclang/noisy/advanced/listref.tex}
%\input{inc/uclang/extra/irgen/listref.tex}
